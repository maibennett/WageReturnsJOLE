\begin{table}[ht]
\caption{Local labor market conditions at various ages}
\label{tab:locallabmktbyage}
\centering
\scalebox{1.0}[1.0]{% 
\begin{threeparttable}
\begin{tabular}{lrrr@{}l}
\toprule 
Variable \phantom{extraspacehere} & NLSY79 & NLSY97 & \multicolumn{2}{c}{97--79} \\
\midrule 
\multicolumn{5}{l}{\emph{County Employment Rate:}} \\
~~At age 16                      & 0.74 & 0.88 & 0.14 & *** \\ 
~~At age 22                      & 0.78 & 0.88 & 0.09 & *** \\ 
~~At age 26                      & 0.83 & 0.88 & 0.04 & *** \\ 
~~At age 29                      & 0.85 & 0.86 & 0.01 & ** \\ 
\vspace{-6pt}  \\
\multicolumn{5}{l}{\emph{County Ave. Income per Worker:}} \\
~~At age 16                      & 12.40 & 16.67 & 4.27 & *** \\ 
~~At age 22                      & 13.39 & 18.13 & 4.73 & *** \\ 
~~At age 26                      & 14.69 & 18.68 & 3.99 & *** \\ 
~~At age 29                      & 15.14 & 18.83 & 3.69 & *** \\ 
\vspace{-6pt}  \\
\multicolumn{5}{l}{\emph{Number of four-year colleges in county (per 100,000 people):}} \\
~~At age 16                      & 2.12 & 1.83 & -0.30 & *** \\ 
\vspace{-6pt}  \\
\multicolumn{5}{l}{\emph{Share of youth with at least one four-year college in county:}} \\
~~At age 16                      & 0.85 & 0.82 & -0.03 & *** \\ 
\vspace{-6pt}  \\
\multicolumn{5}{l}{\emph{Average tuition of state flagship university:}} \\~~At age 16                      & 3.31 & 6.81 & 3.50 & *** \\ 
\bottomrule 
\end{tabular} 
\footnotesize{Notes: Employment rate in the respondent's county of residence at each age is the number of employees reported by employers divided by population. Income per worker is the total wage and salary income of the county (in 1,000's of 1982-84\$) divided by the number of workers. Number of colleges and college tuition are computed as of 1988 and 2005 for the respective NLSY panels. That is, we report college information for years 1988 and 2005 in the youth's county of residence at age 16. Summary statistics weighted by NLSY sampling weights. Significance reported at the 1\% (***), 5\% (**), and 10\% (*) levels.}
\end{threeparttable} 
} 
\end{table} 
